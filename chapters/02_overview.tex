% UNIT : Overview of ML approaches to Cog Neuro 
\chapter{Overview of ML approaches to modeling cognitive neuroscience data}
\label{chap:overview}

In this Chapter we have two papers with cognitive neuroscience models as topic.

\section[Analyzing biological and artificial neural networks]{\textit{Analyzing biological and artificial neural networks: challenges with opportunities for synergy?}\\ \cite{BARRETT201955}}

Deep neural networks brought a revolution in the area of ML, with millions of parameters, no engineered features, and very high performance. We face an an \textbf{analogy with neuroscience}, as both fields need to:
\begin{itemize}
    \item understand how neural networks, consisting of large numbers of interconnected elements, transform representations of stimuli across multiple processing stages to implement a wide range of complex computations and behaviours;
    \item describe and analyze very high dimensional data.
\end{itemize}

The analogies appear in four fields: Receptive fields, Ablation, Dimensionality reduction, and Representational geometries.

\subsection{Receptive fields}

\section[Spatial methods, logical methods and ANNs]{\textit{What does the mind learn? A comparison of human and machine learning representations}\\ \cite{SPICER201997}}

